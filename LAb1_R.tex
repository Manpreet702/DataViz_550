% Options for packages loaded elsewhere
\PassOptionsToPackage{unicode}{hyperref}
\PassOptionsToPackage{hyphens}{url}
%
\documentclass[
]{article}
\usepackage{amsmath,amssymb}
\usepackage{iftex}
\ifPDFTeX
  \usepackage[T1]{fontenc}
  \usepackage[utf8]{inputenc}
  \usepackage{textcomp} % provide euro and other symbols
\else % if luatex or xetex
  \usepackage{unicode-math} % this also loads fontspec
  \defaultfontfeatures{Scale=MatchLowercase}
  \defaultfontfeatures[\rmfamily]{Ligatures=TeX,Scale=1}
\fi
\usepackage{lmodern}
\ifPDFTeX\else
  % xetex/luatex font selection
\fi
% Use upquote if available, for straight quotes in verbatim environments
\IfFileExists{upquote.sty}{\usepackage{upquote}}{}
\IfFileExists{microtype.sty}{% use microtype if available
  \usepackage[]{microtype}
  \UseMicrotypeSet[protrusion]{basicmath} % disable protrusion for tt fonts
}{}
\makeatletter
\@ifundefined{KOMAClassName}{% if non-KOMA class
  \IfFileExists{parskip.sty}{%
    \usepackage{parskip}
  }{% else
    \setlength{\parindent}{0pt}
    \setlength{\parskip}{6pt plus 2pt minus 1pt}}
}{% if KOMA class
  \KOMAoptions{parskip=half}}
\makeatother
\usepackage{xcolor}
\usepackage[margin=1in]{geometry}
\usepackage{color}
\usepackage{fancyvrb}
\newcommand{\VerbBar}{|}
\newcommand{\VERB}{\Verb[commandchars=\\\{\}]}
\DefineVerbatimEnvironment{Highlighting}{Verbatim}{commandchars=\\\{\}}
% Add ',fontsize=\small' for more characters per line
\usepackage{framed}
\definecolor{shadecolor}{RGB}{248,248,248}
\newenvironment{Shaded}{\begin{snugshade}}{\end{snugshade}}
\newcommand{\AlertTok}[1]{\textcolor[rgb]{0.94,0.16,0.16}{#1}}
\newcommand{\AnnotationTok}[1]{\textcolor[rgb]{0.56,0.35,0.01}{\textbf{\textit{#1}}}}
\newcommand{\AttributeTok}[1]{\textcolor[rgb]{0.13,0.29,0.53}{#1}}
\newcommand{\BaseNTok}[1]{\textcolor[rgb]{0.00,0.00,0.81}{#1}}
\newcommand{\BuiltInTok}[1]{#1}
\newcommand{\CharTok}[1]{\textcolor[rgb]{0.31,0.60,0.02}{#1}}
\newcommand{\CommentTok}[1]{\textcolor[rgb]{0.56,0.35,0.01}{\textit{#1}}}
\newcommand{\CommentVarTok}[1]{\textcolor[rgb]{0.56,0.35,0.01}{\textbf{\textit{#1}}}}
\newcommand{\ConstantTok}[1]{\textcolor[rgb]{0.56,0.35,0.01}{#1}}
\newcommand{\ControlFlowTok}[1]{\textcolor[rgb]{0.13,0.29,0.53}{\textbf{#1}}}
\newcommand{\DataTypeTok}[1]{\textcolor[rgb]{0.13,0.29,0.53}{#1}}
\newcommand{\DecValTok}[1]{\textcolor[rgb]{0.00,0.00,0.81}{#1}}
\newcommand{\DocumentationTok}[1]{\textcolor[rgb]{0.56,0.35,0.01}{\textbf{\textit{#1}}}}
\newcommand{\ErrorTok}[1]{\textcolor[rgb]{0.64,0.00,0.00}{\textbf{#1}}}
\newcommand{\ExtensionTok}[1]{#1}
\newcommand{\FloatTok}[1]{\textcolor[rgb]{0.00,0.00,0.81}{#1}}
\newcommand{\FunctionTok}[1]{\textcolor[rgb]{0.13,0.29,0.53}{\textbf{#1}}}
\newcommand{\ImportTok}[1]{#1}
\newcommand{\InformationTok}[1]{\textcolor[rgb]{0.56,0.35,0.01}{\textbf{\textit{#1}}}}
\newcommand{\KeywordTok}[1]{\textcolor[rgb]{0.13,0.29,0.53}{\textbf{#1}}}
\newcommand{\NormalTok}[1]{#1}
\newcommand{\OperatorTok}[1]{\textcolor[rgb]{0.81,0.36,0.00}{\textbf{#1}}}
\newcommand{\OtherTok}[1]{\textcolor[rgb]{0.56,0.35,0.01}{#1}}
\newcommand{\PreprocessorTok}[1]{\textcolor[rgb]{0.56,0.35,0.01}{\textit{#1}}}
\newcommand{\RegionMarkerTok}[1]{#1}
\newcommand{\SpecialCharTok}[1]{\textcolor[rgb]{0.81,0.36,0.00}{\textbf{#1}}}
\newcommand{\SpecialStringTok}[1]{\textcolor[rgb]{0.31,0.60,0.02}{#1}}
\newcommand{\StringTok}[1]{\textcolor[rgb]{0.31,0.60,0.02}{#1}}
\newcommand{\VariableTok}[1]{\textcolor[rgb]{0.00,0.00,0.00}{#1}}
\newcommand{\VerbatimStringTok}[1]{\textcolor[rgb]{0.31,0.60,0.02}{#1}}
\newcommand{\WarningTok}[1]{\textcolor[rgb]{0.56,0.35,0.01}{\textbf{\textit{#1}}}}
\usepackage{graphicx}
\makeatletter
\newsavebox\pandoc@box
\newcommand*\pandocbounded[1]{% scales image to fit in text height/width
  \sbox\pandoc@box{#1}%
  \Gscale@div\@tempa{\textheight}{\dimexpr\ht\pandoc@box+\dp\pandoc@box\relax}%
  \Gscale@div\@tempb{\linewidth}{\wd\pandoc@box}%
  \ifdim\@tempb\p@<\@tempa\p@\let\@tempa\@tempb\fi% select the smaller of both
  \ifdim\@tempa\p@<\p@\scalebox{\@tempa}{\usebox\pandoc@box}%
  \else\usebox{\pandoc@box}%
  \fi%
}
% Set default figure placement to htbp
\def\fps@figure{htbp}
\makeatother
\setlength{\emergencystretch}{3em} % prevent overfull lines
\providecommand{\tightlist}{%
  \setlength{\itemsep}{0pt}\setlength{\parskip}{0pt}}
\setcounter{secnumdepth}{-\maxdimen} % remove section numbering
\usepackage{bookmark}
\IfFileExists{xurl.sty}{\usepackage{xurl}}{} % add URL line breaks if available
\urlstyle{same}
\hypersetup{
  pdftitle={GGLab1},
  pdfauthor={Manpreet Singh},
  hidelinks,
  pdfcreator={LaTeX via pandoc}}

\title{GGLab1}
\author{Manpreet Singh}
\date{2026-01-12}

\begin{document}
\maketitle

\subsection{R Markdown}\label{r-markdown}

This is an R Markdown document. Markdown is a simple formatting syntax
for authoring HTML, PDF, and MS Word documents. For more details on
using R Markdown see \url{http://rmarkdown.rstudio.com}.

When you click the \textbf{Knit} button a document will be generated
that includes both content as well as the output of any embedded R code
chunks within the document. You can embed an R code chunk like this:

\begin{Shaded}
\begin{Highlighting}[]
\FunctionTok{library}\NormalTok{(readr)}
\FunctionTok{library}\NormalTok{(dplyr)}
\end{Highlighting}
\end{Shaded}

\begin{verbatim}
## 
## Attaching package: 'dplyr'
\end{verbatim}

\begin{verbatim}
## The following objects are masked from 'package:stats':
## 
##     filter, lag
\end{verbatim}

\begin{verbatim}
## The following objects are masked from 'package:base':
## 
##     intersect, setdiff, setequal, union
\end{verbatim}

\begin{Shaded}
\begin{Highlighting}[]
\FunctionTok{library}\NormalTok{(ggplot2)}

\CommentTok{\# Read the data}
\NormalTok{gap\_df }\OtherTok{\textless{}{-}} \FunctionTok{read\_csv}\NormalTok{(}\StringTok{"https://raw.githubusercontent.com/UofTCoders/workshops{-}dc{-}py/master/data/processed/world{-}data{-}gapminder.csv"}
\NormalTok{)}
\end{Highlighting}
\end{Shaded}

\begin{verbatim}
## Rows: 38982 Columns: 14
\end{verbatim}

\begin{verbatim}
## -- Column specification --------------------------------------------------------
## Delimiter: ","
## chr  (4): country, region, sub_region, income_group
## dbl (10): year, population, life_expectancy, income, children_per_woman, chi...
## 
## i Use `spec()` to retrieve the full column specification for this data.
## i Specify the column types or set `show_col_types = FALSE` to quiet this message.
\end{verbatim}

\begin{Shaded}
\begin{Highlighting}[]
\FunctionTok{head}\NormalTok{(gap\_df, }\DecValTok{5}\NormalTok{)}
\end{Highlighting}
\end{Shaded}

\begin{verbatim}
## # A tibble: 5 x 14
##   country  year population region sub_region income_group life_expectancy income
##   <chr>   <dbl>      <dbl> <chr>  <chr>      <chr>                  <dbl>  <dbl>
## 1 Afghan~  1800    3280000 Asia   Southern ~ Low                     28.2    603
## 2 Afghan~  1801    3280000 Asia   Southern ~ Low                     28.2    603
## 3 Afghan~  1802    3280000 Asia   Southern ~ Low                     28.2    603
## 4 Afghan~  1803    3280000 Asia   Southern ~ Low                     28.2    603
## 5 Afghan~  1804    3280000 Asia   Southern ~ Low                     28.2    603
## # i 6 more variables: children_per_woman <dbl>, child_mortality <dbl>,
## #   pop_density <dbl>, co2_per_capita <dbl>, years_in_school_men <dbl>,
## #   years_in_school_women <dbl>
\end{verbatim}

\begin{Shaded}
\begin{Highlighting}[]
\CommentTok{\# Filter for year 1862}
\NormalTok{gap\_df }\OtherTok{\textless{}{-}}\NormalTok{ gap\_df }\SpecialCharTok{\%\textgreater{}\%}
  \FunctionTok{filter}\NormalTok{(year }\SpecialCharTok{==}\StringTok{"1862"}\NormalTok{)}

\CommentTok{\# View first 5 rows}
\FunctionTok{head}\NormalTok{(gap\_df, }\DecValTok{5}\NormalTok{)}
\end{Highlighting}
\end{Shaded}

\begin{verbatim}
## # A tibble: 5 x 14
##   country  year population region sub_region income_group life_expectancy income
##   <chr>   <dbl>      <dbl> <chr>  <chr>      <chr>                  <dbl>  <dbl>
## 1 Afghan~  1862    4010000 Asia   Southern ~ Low                     27.6    697
## 2 Albania  1862     564000 Europe Southern ~ Upper middle            35.4    736
## 3 Algeria  1862    3590000 Africa Northern ~ Upper middle            28.8   1160
## 4 Angola   1862    2160000 Africa Sub-Sahar~ Lower middle            27      811
## 5 Antigu~  1862      36300 Ameri~ Latin Ame~ High                    33.5   1010
## # i 6 more variables: children_per_woman <dbl>, child_mortality <dbl>,
## #   pop_density <dbl>, co2_per_capita <dbl>, years_in_school_men <dbl>,
## #   years_in_school_women <dbl>
\end{verbatim}

\begin{Shaded}
\begin{Highlighting}[]
\CommentTok{\# Create the scatter plot}
\FunctionTok{ggplot}\NormalTok{(gap\_df, }\FunctionTok{aes}\NormalTok{(}\AttributeTok{x =}\NormalTok{ children\_per\_woman, }\AttributeTok{y =}\NormalTok{ life\_expectancy)) }\SpecialCharTok{+}
  \FunctionTok{geom\_point}\NormalTok{()}
\end{Highlighting}
\end{Shaded}

\pandocbounded{\includegraphics[keepaspectratio]{LAb1_R_files/figure-latex/q1-1.pdf}}

\subsection{Including Plots}\label{including-plots}

You can also embed plots, for example:

\begin{verbatim}
## -- Attaching core tidyverse packages ------------------------ tidyverse 2.0.0 --
## v forcats   1.0.0     v stringr   1.5.1
## v lubridate 1.9.4     v tibble    3.3.0
## v purrr     1.1.0     v tidyr     1.3.1
## -- Conflicts ------------------------------------------ tidyverse_conflicts() --
## x dplyr::filter() masks stats::filter()
## x dplyr::lag()    masks stats::lag()
## i Use the conflicted package (<http://conflicted.r-lib.org/>) to force all conflicts to become errors
## Rows: 38982 Columns: 14
## -- Column specification --------------------------------------------------------
## Delimiter: ","
## chr  (4): country, region, sub_region, income_group
## dbl (10): year, population, life_expectancy, income, children_per_woman, chi...
## 
## i Use `spec()` to retrieve the full column specification for this data.
## i Specify the column types or set `show_col_types = FALSE` to quiet this message.
\end{verbatim}

\begin{verbatim}
## # A tibble: 6 x 15
##   country  year population region sub_region income_group life_expectancy income
##   <chr>   <dbl>      <dbl> <chr>  <chr>      <chr>                  <dbl>  <dbl>
## 1 Afghan~  1970   11100000 Asia   Southern ~ Low                     45.8   1180
## 2 Afghan~  2015   33700000 Asia   Southern ~ Low                     57.9   1750
## 3 Albania  1970    2150000 Europe Southern ~ Upper middle            67.4   3830
## 4 Albania  2015    2920000 Europe Southern ~ Upper middle            77.6  11000
## 5 Algeria  1970   14600000 Africa Northern ~ Upper middle            57.5   7290
## 6 Algeria  2015   39900000 Africa Northern ~ Upper middle            77.3  13700
## # i 7 more variables: children_per_woman <dbl>, child_mortality <dbl>,
## #   pop_density <dbl>, co2_per_capita <dbl>, years_in_school_men <dbl>,
## #   years_in_school_women <dbl>, years_in_school_ratio <dbl>
\end{verbatim}

\pandocbounded{\includegraphics[keepaspectratio]{LAb1_R_files/figure-latex/q2-1.pdf}}

\begin{verbatim}
## Rows: 38982 Columns: 14
## -- Column specification --------------------------------------------------------
## Delimiter: ","
## chr  (4): country, region, sub_region, income_group
## dbl (10): year, population, life_expectancy, income, children_per_woman, chi...
## 
## i Use `spec()` to retrieve the full column specification for this data.
## i Specify the column types or set `show_col_types = FALSE` to quiet this message.
\end{verbatim}

\begin{verbatim}
## Warning: Removed 1 row containing missing values or values outside the scale range
## (`geom_point()`).
\end{verbatim}

\pandocbounded{\includegraphics[keepaspectratio]{LAb1_R_files/figure-latex/q3-1.pdf}}

\begin{verbatim}
## Rows: 38982 Columns: 14
## -- Column specification --------------------------------------------------------
## Delimiter: ","
## chr  (4): country, region, sub_region, income_group
## dbl (10): year, population, life_expectancy, income, children_per_woman, chi...
## 
## i Use `spec()` to retrieve the full column specification for this data.
## i Specify the column types or set `show_col_types = FALSE` to quiet this message.
\end{verbatim}

\pandocbounded{\includegraphics[keepaspectratio]{LAb1_R_files/figure-latex/q4-1.pdf}}

\begin{verbatim}
## Rows: 38982 Columns: 14
## -- Column specification --------------------------------------------------------
## Delimiter: ","
## chr  (4): country, region, sub_region, income_group
## dbl (10): year, population, life_expectancy, income, children_per_woman, chi...
## 
## i Use `spec()` to retrieve the full column specification for this data.
## i Specify the column types or set `show_col_types = FALSE` to quiet this message.
\end{verbatim}

\pandocbounded{\includegraphics[keepaspectratio]{LAb1_R_files/figure-latex/q5-1.pdf}}

\begin{verbatim}
## Rows: 38982 Columns: 14
## -- Column specification --------------------------------------------------------
## Delimiter: ","
## chr  (4): country, region, sub_region, income_group
## dbl (10): year, population, life_expectancy, income, children_per_woman, chi...
## 
## i Use `spec()` to retrieve the full column specification for this data.
## i Specify the column types or set `show_col_types = FALSE` to quiet this message.
\end{verbatim}

\pandocbounded{\includegraphics[keepaspectratio]{LAb1_R_files/figure-latex/q6-1.pdf}}

\begin{verbatim}
\end{verbatim}

Note that the \texttt{echo\ =\ FALSE} parameter was added to the code
chunk to prevent printing of the R code that generated the plot.

\end{document}
